\documentclass[11pt]{article}

\usepackage[french]{babel}
\usepackage[utf8x]{inputenc}
\usepackage[OT1]{fontenc}
\usepackage{fullpage}
%\usepackage[top=2cm, bottom=1.3cm, left=2cm, right=4cm]{geometry}
\usepackage{hyperref}
\usepackage{marginnote}

%% \usepackage{tensor}
%% \usepackage{enumitem}
%% \usepackage{outlines}
%% \usepackage{ulem}

\usepackage{makeidx}
\usepackage{xparse}

\usepackage{fontawesome5}
\usepackage{awesomebox}
\usepackage{xcolor}
\usepackage[most]{tcolorbox}
\usepackage{amssymb}
\usepackage{soul}

% ****************************************************************************
% ********************************************************************* COMMON 
% ****************************************************************************
%% *****************************************************************************
%% ********************************************************************* questOK
%% *****************************************************************************
\newtcolorbox{questbox}{colback=violet!10,colframe=white,left skip=0.2\textwidth}
\newcommand{\quest}[2]{%
  {\footnotesize
    \begin{questbox}{\hfill\textbf{#1}\vspace{2mm}\hrule}
    \begin{awesomeblock}[violet]{2pt}{\faQuestionCircle[regular]}{violet}
      #2
    \end{awesomeblock}
  \end{questbox}
  }
}

\newtcolorbox{questOKbox}{colback=green!10,colframe=white,segmentation style={draw=violet,solid},left skip=0.2\textwidth}
\newcommand{\questOK}[3]{%
  {\footnotesize
  \begin{questOKbox}{\hfill\textbf{#1}\vspace{2mm}\hrule}
    \begin{awesomeblock}[green]{2pt}{\faQuestionCircle}{green}
      {
        #2

        {\color{green}\vspace{2mm}\hrule\vspace{2mm}}

        $\leadsto$ #3
      }
    \end{awesomeblock}
  \end{questOKbox}
  }
}
%% *****************************************************************************
%% *****************************************************************************
%% *****************************************************************************
\newcommand{\nrj}[1][]{{\color{teal} #1\faBolt}}
\newcommand{\nour}[1][]{{\color{olive} #1\faDrumstickBite}}
\newcommand{\rich}[1][]{{\color{lightgray} #1\faCoins}}
\newcommand{\rep}[1][]{{\color{orange} #1\faGrin}}
\newcommand{\magic}[1][]{{\color{violet} #1\faSun}}
\newcommand{\life}[1][]{{\color{red} #1\faHeart}}
\newcommand{\terror}[1][]{{\color{black} #1\faSurprise}}

\newtcolorbox{tradebox}{colback=yellow!10,colframe=white}
%% *****************************************************************************
%% Define a macro parsing its input argument one character at a time

%% needed in Tex to be used in macro name
\makeatletter

\newcounter{@tradeitems}
%% print ', ' if already an item before
\newcommand\@tradeseparator{%
  \ifnum\value{@tradeitems}>0 {,} \fi
  \addtocounter{@tradeitems}{1}
}

%% e.E: Energie, n/N: NOUR, o/O: RICH, r/R: REP, m/M: MAGIC
%% \trade[option put after]{combinaison de eXnXoXrXmX > EXNXOXRXMX}
\newcommand\trade[2][]{%
  \setcounter{@tradeitems}{0}
  %\marginnote[right]{[\@mytrade#2\@nnil#1]}
  [\@mytrade#2\@nnil#1]
}

\def\@mytrade#1{%
  %% if #1 == \@nil do nothing and stop expanding
  %% sinon utilise \mymacro@char@#1 et continue la suite
  \ifx\@nnil#1\relax\else
  \@nameuse{mytrade@char@#1\expandafter}%
  \fi
}
%% Définit un "faiseur de macro"
%% \defcharcode#1 définira une macro \mytrade@char@#1 qui finira par
%% appeler récursirvement \@mytrade
\def\defcharcode#1{%
    \@namedef{mytrade@char@#1}%
}
%% e: use NRJ
\defcharcode{e}#1{%
  \@tradeseparator%
  \nrj[-#1]%
  \@mytrade
}
%% n: use NOUR
\defcharcode{n}#1{% reads further character after q
  \@tradeseparator%  
  \nour[-#1]%
  \@mytrade
}
%% o: use OR
\defcharcode{o}#1{% reads further character after q
  \@tradeseparator%
  \rich[-#1]%
  \@mytrade
}
%% r: use REP
\defcharcode{r}#1{% reads further character after q
  \@tradeseparator%
  \rep[-#1]%
  \@mytrade
}
%% m: use MAG
\defcharcode{m}#1{% reads further character after q
  \@tradeseparator%
  \magic[-#1]%
  \@mytrade
}
%% v: use LIFE
\defcharcode{v}#1{% reads further character after q
  \@tradeseparator%
  \life[-#1]%
  \@mytrade
}
%% t: use TERROR
\defcharcode{t}#1{% reads further character after q
  \@tradeseparator%
  \terror[#1]%
  \@mytrade
}
\defcharcode{>}{%
  \setcounter{@tradeitems}{0}
   $\Rightarrow$ 
  \@mytrade
}
%% E: gain NRJ
\defcharcode{E}#1{%
  \@tradeseparator%
  \nrj[#1]\index{Gain!Energie}%
  \@mytrade
}
%% N: gain NOUR
\defcharcode{N}#1{% reads further character after q
  \@tradeseparator%
  \nour[#1]\index{Gain!Nourriture}% 
  \@mytrade
}
%% O: gain OR
\defcharcode{O}#1{% reads further character after q
  \@tradeseparator%
  \rich[#1]\index{Gain!Richesse}%
  \@mytrade
}
%% R: gain REP
\defcharcode{R}#1{% reads further character after q
  \@tradeseparator%
  \rep[#1]\index{Gain!Reputation}%
  \@mytrade
}
%% M: gain MAG
\defcharcode{M}#1{% reads further character after q
  \@tradeseparator%
  \magic[#1]\index{Gain!Magie}%
  \@mytrade
}
%% V: gain LIFE
\defcharcode{V}#1{% reads further character after q
  \@tradeseparator%
  \life[#1]\index{Gain!Vie}%
  \@mytrade
}
%% T: gain Terror
\defcharcode{T}#1{% reads further character after q
  \@tradeseparator%
  \terror[-#1]\index{Gain!Terreur}%
  \@mytrade
}


\makeatother
%% *****************************************************************************

%% *****************************************************************************
%% ****************************************************************** secret/loc
%% *****************************************************************************
\newcommand{\loc}[1]{%
  (lieu #1)
}
\newcommand{\secret}[1]{%
  [secret #1]
}
%% using xparse
%% 1=Nom, [2=entry, default=XX], [3=key pour index sans accent]
\NewDocumentCommand{\lieu}{m O{XX} o}{%
  \underline{#1} (#2)\index{\IfNoValueTF{#3}{#1}{#3@#1}}%
}

\def\adiante{\lieu{Mer d'Adiante}[153]}
\def\blanc{\lieu{Blanc-Lichen}[107]}
\def\boismuraille{\lieu{Bois-Muraille}}
\def\bosquet{\lieu{Bosquet des Chasseurs}[102]}
\def\bourgr{\lieu{Bourg Pestiféré}[138,142][Bourg]}
\def\bourgpacif{\lieu{Bourg Pacifié}[142][Bourg]}
\def\broch{\lieu{Broch Cruach}[136]}
\def\bundorca{\lieu{Bundorca}[135,148]}
\def\camelot{\lieu{Camelot}}
\def\cerclelunaire{\lieu{Cercle Lunaire}[133]}
\def\chatoyantes{\lieu{Étendues Chatoyantes}[157][Etendues Chatoyantes]}
\def\conclave{\lieu{Conclave Calciné}[104][Conclave Calcine]}
\def\croisee{\lieu{Croisée}[152][Croisee]}
\def\cuanacht{\lieu{Cuanacht}[101]}
\def\debacle{\lieu{Debâcle}[118][Debacle]}
\def\faldorca{\lieu{Faldorca}[148]}
\def\falfuar{\lieu{Falfuar}[134,147]}
\def\flotte{\lieu{Flotte d'Épaves}[119][Flotte d'Epaves]}
\def\foire{\lieu{Foire du Guerrier}[103]}
\def\hospice{\lieu{Hospice Insulaire}[109]}
\def\lacmiroir{\lieu{Lac Miroir}[113]}
\def\lames{\lieu{Lames de la Mélancholie}[105][Lames de la Melancholie]}
\def\larvebois{\lieu{Larvebois}[108]}
\def\loincomtat{\lieu{Loincomtat}[116]}
\def\newcamelot{\lieu{Nouvelle Camelot}[190]}
\def\marchestitan{\lieu{Marches du Titan}[115]}
\def\monasteremere{\lieu{Monastère de la Toute-Mère}[140][Monastere de la Toute-Mere]}
\def\murmures{\lieu{Forêt des Murmures}[155][Foret des Murmures]}
\def\nidcorbeaux{\lieu{Nid-de-Corbeaux}[160]}
\def\noirbourbe{\lieu{Noirebourbe}[140]}
\def\plainessacrees{\lieu{Plaines Sacrées}[???][Plaines Sacrees]}
\def\premierfort{\lieu{Premier Fort}[117]}
\def\templemere{\lieu{Temple de la Toute-Mère}[115][Temple de la Toute-Mere]}
\def\tombesordre{\lieu{Tombes de l'Ordre}[154]}
\def\tordracine{\lieu{Tordracine}[114]}
\def\tuathan{\lieu{Tuathan}}
\def\tumulus{\lieu{Tumulus des Précurseurs}[106][Tumulus des Precurseurs]}
%%\newcommand{\perso}[1]{\textbf{#1}}
%% using xparse
%% 1=Nom, [2=key pour index sans accent]
\NewDocumentCommand{\perso}{m o}{%
  \textbf{#1}\index{\IfNoValueTF{#2}{#1}{#2@#1}}%
}
\def\agravain{\perso{Agravain}}
\def\amergin{\perso{Amergin}}
\def\arev{\perso{Arev}}
\def\arthur{\perso{Arthur}}
\def\aubert{\perso{Aubert}}
\def\badb{\perso{Badb}}
\def\bedivere{\perso{Bedivère}[Bedivere]}
\def\beor{\perso{Béor}[Beor]}
\def\bors{\perso{Bors}}
\def\breagach{\perso{Bréagach}[Breagach]}
\def\cosuil{\perso{Cosuil}}
\def\dame{\perso{Dame du Lac}}
\def\dobromir{\perso{Dobromir}}
\def\erfir{\perso{Erfyr}}
\def\fael{\perso{Faël}[Fael]}
\def\gaheris{\perso{Gaheris}}
\def\galaad{\perso{Galaad}}
\def\gauvain{\perso{Gauvain}}
\def\geraint{\perso{Geraint}}
\def\guivrarpion{\perso{Guivrarpion}}
\def\kincaid{\perso{Kincaid}}
\def\larve{\perso{Larve}}
\def\lamorak{\perso{Lamorak}}
\def\lancelot{\perso{Lancelot}}
\def\macha{\perso{Macha}}
\def\mordred{\perso{Mordred}}
\def\morfran{\perso{Morfran}}
\def\morgane{\perso{Morgane}}
\def\neante{\perso{Néante}[Neante]}
\def\nemain{\perso{Nemain}}
\def\orrin{\perso{Orrin}}
\def\palamede{\perso{Palamède}[Palamede]}
\def\yvain{\perso{Yvain}}

%\newcommand{\art}[1]{\texttt{#1}}
\NewDocumentCommand{\art}{m o}{%
  \texttt{#1}\index{\IfNoValueTF{#2}{#1}{#2@#1}}%
}
\def\craie{\art{Craie Rituelle}}
\def\graal{\art{Graal}}
\def\fauxgraal{\art{Faux Graal}}
\def\guivre{\art{Guivre}}
\def\masque{\art{Masque Funéraire}[Masque Funeraire]}
\def\outilspierre{\art{Outils de Tailleur de Pierre}}
\def\pierrerancune{\art{Pierre des Rancunes}}
\def\poupee{\art{Poupée de Paille}[Poupee de Paille]}
\def\tetemorrigan{\art{Tête de Morrigan}[Tete de Morrigan]}

\newcommand{\gain}[1]{{\color{red}\textbf{[#1]}}}

\newcommand{\unk}[1]{{\color{orange}(#1)}}



% ****************************************************************************
% ********************************************************************** title
% ****************************************************************************
\title{%
Entre le Wyrd d'Avalon.
}

\makeindex
%% *****************************************************************************
%% ******************************************************************** document
%% *****************************************************************************
\begin{document}

\maketitle

Énergie: \nrj{}; Nourriture: \nour{}; Richesse: \rich{}; Réputation: \rep{}; Magie: \magic{}, Vie: \life{}, Terreur: \terror{}.

% ****************************************************************************
% ***************************************************************** Chapitre 1 
% ****************************************************************************
\section{Chapitre 1}

Nous quittons le village de \cuanacht{} \trade{e1>R1} pour essayer de comprendre ce qu'il est advenu de \neante{}, \yvain{}, \aubert{}, \erfir{} et \fael{}. Aux \lames{} \trade[, obj SI Forgeron]{e1r1>}\index{Gain!Objet}, un forgeron que connaît \beor{} manie un marteau remarquable que nous ne manquons pas d'acquérir. En traînant \larve{} du côté des restes calcinés des druides, au \conclave{}, on acquiert les «Rites du Menhir» \secret{111}. Et on apprend que la première expédition serait partie vers \tuathan{}. \gain{+1XP}.

% ****************************************************************************
% ***************************************************************** Chapitre 2 
% ****************************************************************************
\section{Chapitre 2}


On chemine ensuite vers \blanc{} \trade{e1n1>R1} où du lichen (cristaux de sel) disparaît. On traverse \tordracine{} \trade[, r.VERT]{e2>N2}où on trouve une loge de gardiens-pisteurs, ainsi que le corps de \fael{}, la poitrine percée d'une lance de wyrdacier. Dans sa sacoche, le \fauxgraal{} et un mot indiquant qu'ils se sont séparé, pour augmenter les chances d'aller à \newcamelot{}.
\questOK{Femme de \fael{}}{Aller à \blanc{} pour donner à la femme de \fael{} des nouvelles de son funeste destin.}{On apprend que \fael{} était \galaad{}.}

Nous sauvons aussi \cosuil{}, une femme qui cherchait un érudit (il vivrait à \boismuraille{}) permettant de comprendre ses notes familiales sur la façon de vaincre les Précurseurs.

Plus loin, à \lacmiroir{} \trade{>E1T1V1}, nous tentons sans succès d'invoquer la \dame{}, mais nous ne sommes sans doute pas assez importants pour elle.

Aux \marchestitan{}, au \templemere{}, \gauvain{} est blessé et la foule, contaminée par la mort rouge, veut le déloger du temple. Il nous confie le fourreau d'excalibur.

\questOK{Fourreau d'Excalibur}{Confié par \gauvain{} qui nous charge d'aller le ramener dans les Tombes de l'Ordre. \gain{+2XP, 4° Restaurer l'Ordre}}{On le fait, ce qui contrarie \palamede{} et \bedivere{}.}

\quest{Sort des Ermites}{Que sont devenus les Ermites du \templemere{} qui ont disparu ?}

Un \bourgr{} proche est le siège de tumultes et de rébellion. Une vielle femme qui porte le collier de \neante{} nous apprend qu'ils sont sont passés deux fois : 5 il y a trois semaine dans un sens, et 2 (\neante{} et \erfir{}) il y a une semaine. Ils avaient perdus l'un des leurs à \tordracine{}.

\questOK{Perdu à \tordracine{}}{Retrouver le membre de la première expédition qui est mort au fond d'une gorge à \tordracine{}}{C'est clairement \fael{}}.

Nous choisissons d'aider \gaheris{}, le chevalier qui a été mandé par \morgane{} pur rétablir l'ordre. Cela se termine par une bataille sanglante qui ramène une sorte de paix, d'où le nouveau nom de \bourgpacif{}.

Un peu au sud, au \premierfort{}, un camp récent mais dévasté, un vaste trou terrifiant (il abriterait une \guivre{}). Quelqu'un, venu de Camelot, a embauché \erfir{} pour y réparer le Menhir. A peine à l'est, le marécage de \noirbourbe{} \trade{e2v1>M1} aurait abrité jadis le \monasteremere{} mais il faut de la richesse ou de la nourriture pour explorer plus avant. Plus au sud, la \flotte{} \trade[, 1XP]{e4t1>M1}\index{Gain!Expérience@Experience} abrite les reste de la flotte qui permit à Arthur de gagner Avalon. La flotte, qui pourrait permettre de quitter l'île, est protégée par un dôme invisible.

\questOK{Flotte oubliée du Temps}{Se renseigner au \cerclelunaire{} sur la flotte protégée par la magie des Druides.}{C'est Merlin qui a protégé les navire qui pourront peut-être servir à partir.}

Et on peut enfin arriver jusqu'à \newcamelot{} \trade[1 XP]{e2>}\index{Gain!Expérience@Experience} qui a subit les ravages du temps. Le Roi Arthur est mort. Nous parlons à \lancelot{}, qui a bien besoin d'aide. Les agissements de \gaheris{} le peinent, mais «on fait pour le mieux»... Les problèmes actuels:
\begin{itemize}
\item à l'Ouest: \bors{} a disparu, réglait conflit entre \nidcorbeaux{} et \boismuraille{}.
\item au Nord: \bedivere{} et \palamede{} sont en mission secrète, \gauvain{} envoyé en renfort.
\item \agravain{} rendu aux portes de \tuathan{} pour confirmer retour des Précurseurs.
\item \geraint{} est mort, on attend que les Druides du \cerclelunaire{} élisent son remplaçant.
\item \lamorak{} est parti enquêter à l'Ouest sur la menace du Wyrd.
\item \galaad{} a disparu depuis longtemps.
\end{itemize}

\quest{Reformer la Table Ronde}{Il faut pour cela aider les Chevalier et avancer sur les statuts \texttt{Restaurer l'Ordre} (en avoir 6/8).}

OU BIEN

\quest{Fin de la Table Ronde}{Empêcher les Chevaliers de revenir, liée aux statuts \texttt{Chute de la Chevalerie} (en avoir 6/8).}

Nous avons un peu plus de succès avec \morgane{} qui pourrait nous aider si nous allons chercher pour elle un souvenir (une poupée) sur la plage des \chatoyantes{}. \secret{475} \gain{+2XP}.

\questOK{Tâche pour Morgane}{Aller chercher un souvenir d'enfance dans un tour de la plages des \chatoyantes{}.}{Fait}

% ****************************************************************************
% **************************************************************** Chapitre 3A 
% ****************************************************************************
\section{Chapitre 3A}
Il s'agit essentiellement d'un aller-retour, en passant par \bourgpacif{}, les \marchestitan{}, \tordracine{} avant de descendre vers \blanc{} et vers le levant par le \premierfort{}, \noirbourbe{} et enfin les \chatoyantes{} \trade{e2>T1}. Sur place, on explore le sommet d'une tour branlante malgré l'attrait des sillons étranges dans le sable. Un coffret contient une \poupee{}. \gain{2XP, «Bien Précieux»}. Ici, on rêve d'un masque à trois yeux.

\questOK{Rapporter à Morgane}{Il faut maintenant rapporter la poupée à \morgane{}}{Fait en retournant à \newcamelot{}}

Et nous voici de nouveau à \newcamelot{}. Entrevue avec \morgane, qui reprend la \poupee{} et nous demande de l'aider, nous acceptons. \secret{417} Elle nous confie alors le \fauxgraal{}, en nous disans de nous renseigner auprès de la \dame{}, ou auprès des druides du \cerclelunaire{} en particulier à \amergin{}. \gain{1XP, \magic[1]\index{Gain!Magie}, «Mission au Cercle Lunaire»}.

% ****************************************************************************
% **************************************************************** Chapitre 4A 
% ****************************************************************************
\section{Chapitre 4A}

On part vers le nord du \bourgpacif{}, vers \broch{} \trade[, r.VIO]{e2>O1}. Rencontres avec des voyageurs oniriques puissants. A force de tourner, nous y gagnons \gain{«5° Heureuse Rencontres»}, un Nain que nous écoutons longtemps \secret{238} \gain{1XP, \nour[-2], \terror[3]}. On peut aussi en apprendre plus sur le forgeron \morfran{} qui aurait forgé des chaînes pour les esclavagistes \gain{«3° Successeur de Geraint»}.

A peine à l'Ouest, les deux villages de \bundorca{} et \falfuar{} se partagent un Menhir depuis des temps immémoriaux. Nous aidons à rapatrier tout le monde à Bundorca qui devient \faldorca{} tandis que Falfuar tombe à l'abandon, ce qui crée une famine en attendant qu'on trouve un meilleur havre pour ces gens. N'empêche, on peut toujours aller à la taverne \trade{o1>T2} ou faire des offrandes \trade[si inf 2]{t1>M1}.

Au nord, les \tombesordre{} nous attendent. On effectue une prudente descente dans le puit imposant sans rien trouver d'autre que des ossements, une épée brisée \rich[1]. Il y a aussi \bedivere{} et \palamede{} qui jouent aux cartes et qui sont contrariés que \emph{nous} ramenions le fourraux d'Excalibur \gain{2XP, \rich[4]}.

\questOK{Faire parler les Chevaliers}{Pour savoir ce qui s'est passé (ici aux \tombesordre{}). Ils sont tremblants et refusent de parler}{Ils ont essayé de rescusciter \arthur{} avec \neante{} et \graal{}.}

Un petit passage dans la \murmures{} \trade[, r.VERT]{e2>N2}, essentiellement pour chasser. Il faut commencer à entrer dans la légende (hautes caractéristiques) pour découvrir ses secrets. (PRAGMATISME 3 nous ouvre \gain{\terror[-1], «1° Secret de la Forêt»}.

Il est temps d'aller voir à l'Ouest. On repasse par \falfuar{}, maintenant à l'abandon. Il s'y cache un passage bien dissimulé vers \tordracine{} \nrj[-2], ce qui sera pratique quand le Précurseur qui nous poursuit parfois se montre. Il ne peut pas nous y suivre. Il est possible d'y chasser (r.VERT), ou de piller.

Ensuite nous arrivons au \cerclelunaire{} \trade{e2o1>M1} ou \trade{e2n1>M1}, lieu sacré pour les druides, mais un peu désert. Un vieillard nous explique que l'arbre étrange qui protège la première flotte à la \flotte{} a été érigé par Merlin lui-même, en attendant le temps où les navires pourront être réutilisés \gain{«8° Eclaireur»}.

\quest{Contacter \orrin{} à propos de ces Navires}{\orrin{} pourrait peut-être lever la protection sur les Navires de la \flotte{}, ce qui pourrait permettre de repartir.}

Dans les masures, un partie du groupe \hl{sans \larve{}} peut en apprendre plus sur le problème de la succession de \geraint{}. Un jeune, \breagach{}, et un vieux forgeron, \morfran{}, sont candidats et on peut en savoir plus sur eux à \broch{} ou à \blanc{}, \gain{«1° Successeur de Geraint»}.

\begin{itemize}
\item \morfran{} se vante d'avoir aidé des gens à échappper à l'esclavage à \broch{}.
\item \breagach{} a soigné des gens à \blanc{}.
\end{itemize}

\questOK{Succession de Geraint}{En apprendre plus sur les deux candidats.}{\morfran{} a aidé les esclavagistes, \breagach{} n'ai aidé personne et exilé.}

Nous rendons visite à l'archidruide \amergin{}, prophète intellectuellement limité. Il révèle notre \fauxgraal{} qui se transforme en boue, mais le vrai serait tout près, dans la \adiante{}. \gain{1XP, \magic[1]}.

% ****************************************************************************
% ***************************************************************** Chapitre 5 
% ****************************************************************************
\section{Chapitre 5}

Où chercher ? Passage à la \croisee{}, 4 routes et une \pierrerancune{}. En essayant de faire du troc (\nrj[-1]), rencontre vieil historien \secret{172} qui s'intéresse à la \pierrerancune{}. Une vieille légende nous semble avoir été vécue... par nous! \gain{\rich[1], 2XP, «7° Mystère Brûlant», «5° Main Secourable»}.

Mais on ne s'arrête pas et on pousse jusqu'à \nidcorbeaux{} \trade{v1>O1}, aux coutumes sombres et dérangeantes. Au marché, \hl{sans \arev{}}, on peut acheter du poisson \trade{o1>N1}. \mordred{} n'est pas disponible pour l'instant. On n'apprend presque rien sur la \tetemorrigan{}, si ce n'est que se sont ses trois aspects qui sont vénérés ici : \nemain{} Dame de la Guerre et du Carnage, \badb{} Dame des Corbeaux et des Prophéties et \macha{}, Dame des Rois et des Pouvoirs Terrestres.

Une première expédition vers la \adiante{} tourne plutôt mal. Avancer à l'aveuglette ne conduit qu'à une mort presque certaine. Mais les rêves permettent de savoir qu'il faut éviter la puanteur et le froid, avancer dans l'obscurité, échapper à un nuage de poussière en glissant le long d'une étrange langue. Le calice est en direction des flots mugissants...

Il est temps d'explorer à nouveau la \adiante{}. On y trouve une tour penchée avec un crâne écarlate qui indique la mort rouge \gain{1XP, «5° Tréfond»}, un vieux tonnelet rempli de pièces \gain{1XP,\rich[5], «2° Tréfond»} et le \graal{} \gain{3XP, \magic[3], «3° Tréfond»}, une bataille fait rage pour nous le prendre. La \dame{}, blessée, nous protège en nous rendant invisibles. Nous la défendons contre des soudards. \gain{1XP,\rich[1]}

\quest{Amener \graal{} à \tuathan{}}{
  \begin{itemize}
  \item Demander de l'aide à \orrin{} qui est au sud de \loincomtat{}, à l'ouest des plaines de \debacle{}.
  \item Retrouver \neante{} dans la \murmures{}.
  \end{itemize}
}

\quest{Relique des Précurseurs}{Retrouver le \masque{} dans les \chatoyantes{}.}

% ****************************************************************************
% ***************************************************************** Chapitre 6 
% ****************************************************************************
\section{Chapitre 6}

Dans le but d'aller à la \murmures{}, nous repassons par \faldorca{} et sa taverne \trade{o1>T2}. Dans la nuit, l'impression d'une grande bataille silencieuse où le Traqueur est vaincu et les gagnants vont vers \tuathan{}, à l'Ouest. Aux \tombesordre{}, nous descendons pour constater que la Tombe d'\arthur{} a été profannée, \gain{1XP, «2° Rêves et Prophéties», «2° Renseignements Troublants», \terror[1]}. Confrontés, \bedivere{} et \palamede{} admettent qu'ils ont essayé de rescusciter les restes d'\arthur{} en utilisant le \graal{} avec \neante{}. On essaye de les remettre sur le droit chemin \gain{«3° Restaurer l'Ordre»}.

A \murmures{}, notre courage nous permet d'arriver àun cheval blanc tracé à la craie \gain{ «2° Secrets de la Forêt», \craie{}}. Puis, la spiritualité nous guide vers une méditation sereine \gain{ «3° Secret de la Forêt», \life[1]}. Plus loin, un château de pierre blanches qui ne cesse de nous échapper (car pas «Vent du Wyrd» et pas de Menhir). Mais toujours pas de nouvelles de \neante{}, on décide donc d'essayer de trouver \orrin{}. Cap au sud.

Nous gagnons rapidement \blanc{}. La femme de \fael{} nous apprend que ce dernier était en fait \galaad{}, elle ne veut plus en entendre parler et nous laisse ses affaires (bouclier, armure) \gain{ «2° Trésors Cachés»}. Remis au seigneur du village qui, abasourdi, nous demande de quitter discrètement le village \gain{ \rich[1], «1° Restaurer l'Ordre»}. On prend quand même le temps de s'intéresser au passé de \breagach{}, pour apprendre qu'il n'a aidé personne et juste fait de l'argent en vendant le lichen comme remède médicinal; il a été exilé \gain{ «2° Successeur de Geraint»}.

De là, direction l'est vers le \bosquet{} \trade[, r.VERT]{e2>N2} et son crâne gigantesque pour d'anciennes offrandes. On ne préfère pas le profaner ou le piller. Cette fois, y rêver ne nous permet pas de courir ``vraiment'' la forêt pour la nourriture (mais... peut-être humain... ancien gain \gain{«1° Marque du Chasseur», «1° Chant du Deuil»}.

Après un temps qui paraît infini, nous voici de nouveau à \cuanacht{}. Sans but, on erre dans les ruelles envahies de Wyrd, on se croirait 600 ans en arrière \gain{\terror[3], \magic[1]}. Un cauchemar nocturne montre une \guivre{} de la taille d'une montagne et des villageois déçus de ne pas avoir été protégés par nous.

Détour rapide par le \lacmiroir{}, on ne cherche pas à y voir la \dame{} (pas «Traqués», pas de Talisman, plus de \fauxgraal{}) et on va chasser (\trade{r1t1>N2}) ce qui déplaît aux esprits.

Plus au sud-ouest, \beor{} veut absolument aller à la \foire{} \trade[,1XP]{e4v1>}pour participer aux tournois qu'il parvient à remporter \gain{1XP, \rep[2]}. Si cela lui attire la célébrité, la rumeur annonce rapidement que nous portons malheur et une voyante annonce une destinée terrifiante \gain{ 3XP, «1° Mystère Brûlant», «Parjure»} (\beor{}).

\beor{} blessé, nous traversons vers l'\hospice{} \trade{e1o1>V3}. Sur place, on se faufile dans les profondeurs interdites sans se faire repére. L'archidruide \guivrarpion{}, enfermé là, il révèle le secret des Menhirs \gain{2XP, «5° Rêves et Prophéties»} \hl{si \larve{} n'est pas l\`a}.

Nouvelle traversée de l'île pour revenir au delà du \bosquet{}, vers le l'ancien \tumulus{} \trade[, 6 sur D6?]{e2t1>O1}, miné par les chasseurs de trésors. On s'enfonce dans les profondeurs pour en atteindre le c\oe{}ur, entrée en vieillacier que l'on force et qui aurait pu permettre d'apprendre le secret des Menhir \gain{\magic[1]\index{Gain!Magie}}.

On arrive enfin à \loincomtat{} \trade{o1>M1}, au pied des murailles infranchissables de \tuathan{}. Une visite au hall nous apprend que, pour l'instant (Ennemis ou Alliés d'Avalon n'est pas décidé), dame \kincaid{} ne reçoit pas. Au marché, les prix se sont littéralement envolés, notamment pour Maître \dobromir{} le fabricant de jouets : \trade{o3>N1}, \trade{n2>O1}, \trade{o3>V4}, \trade[obj]{o3>}, \trade[obj]{>O1}. En quittant le marché, une vieille nous accoste et comme nous connaissons le «Secret du Vieilacier», on repère chez elle des \gain{\outilspierre} pour \rich[2]. On grimpe sur le Dolmen, pour voir l'ancienne capital de \tuathan{} envahie par le Wyrd. Les gardes parlent d'une bataille qui aurait eu lieu au sud.

\quest{Bataille Magique?}{Aller dans \plainessacrees{} au Sud pour en savoir plus sur bataille.}

Le lendemain, on arrive dans les plaines de \debacle{} \trade{e1n1>R1} et leurs gigantesques figures de calcaire, les réfugiés venus de l'Ouest abondent. On trouve une tombe, celle du Seigneur \yvain{} dont le sacrifice touche profondément \larve{} qui pourtant ne l'aimait guère \gain{1 Empathie, «4° leçon finale»}. Nous conseillons aux réfugiés qui demandent de l'aide de se diriger vers \newcamelot{} \gain{«2° rencontres étranges»}. \larve{} reste et rève qu'il senvole \gain{\nrj[1]\index{Gain!Energie}}.

\beor{}, lui,  retourne à la \foire{}. Il prête une main secourable \gain{1XP, «1° main secourable»} à un guerrier en rachetant son contrat. Il se renseigne sur la première expédition et apprend qu'ils sont partis aider au siège de \boismuraille{}, y cherchant un passage vers \tuathan{} mais en sont revenu hagards, sous le choc de la mort d'\yvain{}. \gain{«6° destin expédition»}.

\quest{Retrouver l'expédition}{Il faut se dépêcher de retrouver la première expédition avant qu'il ne soit trop tard.}

Ensemble, nous parvenons \sousmuraille{} pour chasser \trade[, r. Verte]{e2>N1}. En l'observant, nous remarquons une silhouette qui nous fait penser qu'elle n'est pas abandonnée \gain{\nrj[1]\index{Gain!Energie}}. Des mercenaires nous accueillent et voudraient troquer le butins qu'ils ont pillés sur des corps, nous préférons les aider à pousser leurs chariots. En échange, ils nous racontent le mariage de dame \kincaid{} et de \ultan{} le riche marchand de \boismuraille. Avec l'argent de son époux, la dame se refait une santée et vient faire le siège de \boismuraille{}, aidée par la première expédition, mais \yvain{} y change soudain d'allégeance \gain{\nour[1], «4° destin de l'expédition»}. En cherchant à traverser la muraille d'os, on trouve 5 cadavres avec, entre autre, un \talismanterni{} fait dans un étrange matériau, il n'a pas été façonné de la main de l'homme \gain{«3° rêves et prophéties», 1XP}. Enfin, nous arrivons au somment ce qui nous permet d'y pisser fièrement, tels de véritables aventuriers \gain{1XP, \terror[-1], «4 rencontres étranges»}.

Chargés de nourriture, nous repartons vers le sud en repassant par \debacle{} pour arriver à \bordwyrd{} \trade{o2>V3} ou \trade{m2>V3}. Le crannoge central, au bout du pont, est la demeure d'\orrin{}. Il nous trouve d'abord trop impudents de nous considérer comme des Héros (\secret{283}, \rep[-2]), puis trop ambitieux de vouloire sauver Avalon (\secret{483}, \rep[-2]) mais quand on demande ses conseils (\secret{436}), il finit par accepter de nous inviter à séjourner avec lui \gain{2XP, «2° alliés Avalon»}.

Dans son crannoge, il ne nous raconte pas encore comment il a été en Tuathan (il faudrait le «blason de l'ordre» ou «la main du peuple» ou tout savoir sur l'expédition). Il nous apprend cependant que seul \arthur{} pourrait libérer la \flotte{}. Quant à la première expédition, ils ont choisis le pire chemin vers Tuathan à travers la \valleegardiens{} et l'entrée secrète dans la ville haute de \boismuraille{}, alors qu'il y a tellement plus simple... \gain{3XP, «2° destin de l'expédition», «3° destin de l'expédition»}.

\quest{Accéder au mouillage}{Le secret du «mouillage» de la \flotte{} a été perdu avec la mort d'\arthur{}}.

Il accepte finalement de nous aider à aller à Tuathan si nous faisons nos preuves, en nous rappelant que \lancelot{} à \newcamelot{} sait où sont les chevaliers et que la \tetesereine{} permet de revivre des éléments du passé.

% ****************************************************************************
% ***************************************************************** Chapitre 7 
% ****************************************************************************
\section{Chapitre 7}

Nous retournons d'abord à \sousmuraille{} pour continuer au nord vers \boismuraille{}. Nous ne sommes pas «Égarés et déchus» et accédons au pont-levis pour constater que la ville est assiégée. Des mercenaires nous accostent, on demande des infos (3) et ils nous prennent pour des espions, mais nous balayons ces fausses accusations et apprenons que l'armée de \nidcorbeaux{} (de \mordred{}) en fait le siège car les marchands ne les ont pas payé après leur aide à \loincomtat{} \gain{«1° Guerre pour Avalon»}. Notre réputation nous permet d'entrer dans la Cité (7) sans parler à leur chef.

Nous nous renseignons sur la 1ère expédition (2), un poissonnier nous dit qu'ils sont arrivés avec les envahisseurs venus du Nord mais ont escaladé la muraille pour jurer fidélité à \ultan{}. Sont ensuite entré dans la Halle du Roi et ont disparu. \gain{\nrj[-1], «1° et 4° Destin de l'expédition»}. Quand nous essayons à notre tour d'entrer dans la Halle (9) alors que nous ne sommes pas des «Invités d'Honneur», la tension monte \gain{\rep{-1} \life[-1] par pt Aggressivité}. On repart sans avoir parlé à \ultan{} ou avoir visité le quartier commerçant.

On va alors à l'est vers \visageserein{} \trade{e3>T2M1} pour y activer le Menhir, ce qui demande «Mystère résolu». On commence par escalader la tête (8) ce qui permet d'avoir une idée de l'ampleur de la tâche \gain{\terror[-1]}.

\questOK{Raviver le Menhir de Visage Serein}{}{Il faut explorer en passant par l'oreille}.

Dans l'oreille du visage, une corde indique que quelqu'un est passé récemment (6) et on suit sa piste (12), puis \secret{142} pour arriver dans les tréfonds du labyrinthe. Il faut du temps pour rassembler les indices \secret{114}. La salle, avec vieux piège, corps momifié et symboles sybillins sur les mur, est \emph{l'intérieur} d'un Menhir, ce qui est surprenant car elle prédate \arthur{} qui est sensé les avoir construits \gain{«6° Mystère Brûlant», «Mystère Résolu»}.

Un détour par \sousmuraille{} pour chasser un peu et nous repartons vers le nord pour arriver à \longcairn{} \trade[-1 obj]{e1>O1R1}. C'est une ancienne tour de guet érigée après le siège raté de \tuathan{} la cité des Précurseurs. Plutôt que de nous recueillir ou chercher des infos sur \tuathan{} ou chercher \agravain{} on payepour avoir des nouvelles de la 1ère expédition \gain{\rich[-2], «2° Destin de l'Expédition»} qui a fait une courte halte ici avant de continuer dans la vallée, au mépris des avertissements. Arrivés à 5 ils sont repartis à 4.

Et donc, nous nous enfonçons dans la \valleegardiens{}, jonchée d'ossements. La peste sévit dans le village côtier et nous y recroisons le chevalier qui avait vendu son contrat à \beor{} (1 puis \secret{331}). Avec son épouse, ils ont propagé la peste en essayant de retrouver leurs enfants (accusés \secret{246}). Ils ne bougeront plus d'ici et on les laisse tranquilles \gain{+1XP, \terror[1] par pt de Prudence}.

Dans la vallée, on progresse malgré les statues (6)-(8)-(10). La douleur et la peur ont raison de nous, on tombe à côté du cadavre d'\aubert{} \gain{+2XP, «2° Destin Expédition», }. On repart se réfugier dans un trou \secret{413} où sont entassés des voyageurs, depuis de nombreuses années pour certains. C'est plutôt la honte et l'aspect communautaire qui les retiennent ici \secret{441}, \secret{169}.

De retour à \longcairn{}, on cherche le chevalier \agravain{} qui patrouille aux abords.Il veut capturer une créature des Précurseurs et nous l'y aidons \secret{551}. Après avoir vaincu un Conquéreur, ce dernier reste malheureusement muet mais nous arrivons à convaincre \agravain{} de rentrer à \camelot{} \gain{«5° Restaure Ordre»}. 


Nous retournons ensuite à \loincomtat{} en passant chasser \sousmuraille{}.  De là, plein est, nous gagnons \devastation{} \trade{e2v2>O3}, une terre tourmentée. Dans les mines de minéraux d'outre monde on arrive à des salles très anciennes mais encore mystérieuses pour nous (seulement Chap. 10), en attendant \gain{\magic[1], «7° Mystère Brûlant»}. Au niveau des habitations, anciennes et effondrées, un homme momifié est incrusté dans un mûr mais encore ``vivant''. A l'écouter, on sombre presque dans la folie et les étoiles anciennes \gain{«4° Mystère Brûlant», «7° Rêves et Prophéties», \magic[4] par \terror[1] jusqu'à 4}.

Ensuite, descente vers le sud pour gagner \cornes{} en prenant le temps d'activer le menhir de \debacle{} et du \tumulus{}. Deux tours jumelles sont érigées sur des ilôts, face à la falaise. Dans celle avec de la lumière, nous parlons à des veilleurs (18), et leur racontons une histoire courte mais inquiétante (13), ils nous offrent du matériel (16) \gain{\nour[3], \rich[1], «2° Heureuse Rencontres»}. Le plus jeune nous raccompagne en nous demandant d'aller voir dans l'autre tour (11) \gain{\rich[1], «3° Main Secourable»}. Accéder à l'autre tour est dangereux, nous chutons à plusieurs reprises. Comme indiqués dans nos rêves, nous parvenons difficilement à maîtriser puis libérer une jeune femme redevenue sauvage (4,9,15,6,3,21) \gain{«3° Renseignments Troublants», +1XP}. Elle s'enfuit en courant sans demander son reste.

Convaincu que nous n'aurons pas le temps de pousser jusqu'à \larvebois{} avant que le Wyrd ne nous prenne, on repart activer le menhir de \bordwyrd{} afin de pousser vers l'ouest de \devastation{}. On peut ainsi atteindre \ondechute{} où de gigantesques visages ornent la falaise, des cascades se jettant dans la mer. Ou fouille d'abord la maison en ruine (2), un ancien poste de péage \secret{284} avec une trappe qui recèles des trésors \gain{1 objet, équipement aventuriers, \rich[1], «7° Trésors Cachés», \secret{2}:«découverte troublante»}.

La descente vers les visages permet lentement mais sûrement d'atteindre un tunnel glacial (7,10), avec des hurlements terribles. On dégage les décombres au bout et \secret{176} ce qui fait jaillir une rivière souterraine. La progression est encore plus périlleuse et plus lente mais \secret{605} on débouche finalement à \tuathan{}. \gain{\secret{67}:«repos troublé»}.


% ****************************************************************************
% ***************************************************************** Chapitre 8 
% ****************************************************************************
\section{Chapitre 8}

\tuathan{} est un lieu bien étrange et les lois du monde sont parfois changeantes.

% *************************
\subsection{Terreur et Vie}
Lors de cette étape, lors du repos, les pertes de \life{} et \terror{} sont inversées. On évite d'abord (3-2:15) des murs qui se rapprochent, une fontaine qui crache du goudron acide pour échapper in-extremis à un piège (35). Mais on comprend mieux la ville \gain{\nrj[-2], \life[-1], \texttt{quête}}. Puis, (1-2:16) on grimpe un escalier en voyant notre dos devant nous ce qui terrifie ceux moins habitué à la magie \gain{\texttt{quête}, \magic[1] si 3 en Spiritualité}. Enfin, une grande avenue défiant les lois physiques nous vide de notre énergie mais participe à notre édification \gain{\nrj[-4], \texttt{quête}}.

% *********************************
\subsection{Nourriture et Richesse}
Maintenant, il semble qu'il faille se nourrir d'or à la place de nourriture. Après un combat (2-5:V), on arrive dans un labyrinthe (4-1:13) dont nous ne sortons qu'après de longues heures prudentes \gain{\nrj[-2], \texttt{quête}}. Cela dit, nous parfenons ainsi à arriver à un lieu plus plaisant, presque accueillant.

% *****************************
\subsection{Résolution d'Orrin}
La \resolutionorrin{} est un bâtiment solitaire perché sur 4 pattes. Tout est pensé pour bien accueillir les voyageurs, même si un autel sanglant est un peu impressionnant. En fouillant (5) \gain{\nour[4], \rich[2], 2 objets, «3° Rencontres Étranges» } et en se reposant (6) on se sent plus a même de repartir affronter la ville \gain{\life[5], \terror[-5]}.

% ***********************************
\subsection{Epuisé ou la vie s'en va}
Dans le prochain quartier, si le soir ne nous trouve pas épuisé, notre force vitale est aspirée au loin. De nouveau dans les rues de la ville, le ciel ressemble à un vitrail avec un soleil gigantesque (4-2:44), l'ombre des pyramides est propices à repenser au bon temps passé ``avant'' \gain{\life[-3], \terror[-2], \texttt{quête}}. On combat à nouveau, mais on cherche surtout de l'eau. Des chaudrons dans une bâtisse (3-3:25) sont malheureusement remplis d'acide en ébulition. On y jette de l'or et on y insuffle de la magie \secret{67} ce qui fait apparaître un enfant en or qui hurle horriblement \gain{\terror[3], \texttt{quête}}.

% ********************************
\subsection{Faiblesses exacerbées}
Plus loin, nos propres faiblesses sembles exacerbées par Tuathan. Le passé qui nous rattrape. On arrive rapidement à un temple immense (3-1:41), comme une gigantesque cage toracique avec un trône en son centre. Nous devons combattre un Esprit Précurseur avant de pouvoir examiner le trône couvert d'épines en wyrdroche. \ailei{} s'y asseoit et son esprit est aspirés dans les méandres de la Cité \gain{ \life[-4], \terror[3], \magic[6], 2XP, 2 \texttt{quêtes}, «1° Exploration de Tuathan»}.

% ****************
\subsection{Folie oubliée}
Avec ce que nous savons maintenant, nous pouvons continuer l'exploration et arriver dans un lieu étrange, stase et transes nous débarasse de notre peur de la folie. Nous voilà bientôt entourés d'étrange statues d'argile sans visage (5-3:47). Quand elles commencent à communiquer par des ``moi ?'' (39), nos tentatives pour en savoir plus les transforment en nous-même. Et elles attaquent \gain{\texttt{quête}}. Un autre combat (V) et nous arrivons à une petite forge dont le sol est couvert de tâches de sang (6-1:60). Attaqué par un ``chevalier errant'', on profite de l'occasion pour forger des rubis avec notre propre sang \secret{462}, \gain{\terror[1], \rich[6], \texttt{quête}}. Une volonté trop marquée d'explorer nous amène ensuite dans une champignonière (2Q-55) où l'on peut se reposer (63) \gain{\nrj[1]} et dormir \secret{415} \gain{\texttt{quête}}. Toujours plus loin, (3-1:10) un endroit où les choses et les êtres se réparent \gain{\life[1]}, mais il ne faut pas s'attarder car finalement l'endroit s'effondre \secret{475} \gain{\life[-2], \terror[2], \nour[-1] par membre du groupe, \texttt{quête}}. Quand on reprend nos esprits, on parvient enfin à gagner le \coeurtuathan{}.

% ******************************
\subsection{C\oe{}ur de Tuathan}
Cela ressemble en tous points à Camelot, mais inversée. En s'approchant des portes, il y a des traces de combat avec des Précurseurs. Une fois entrés dans la citadelle, nous gagnons d'abord les appartements (2) en passant par le laboratoire de Morgane, mais mieux rangé et avec des livres qui fondent quand on tente de les consulter. Plus loin dans les couloirs (4), nous parvenons à la salle du trône, vide, avec un emplacement pour le graal au dessus du trône. On y dépose le Graal, se qui nous soulage. Dans la salle de la table ronde (7), il n'y qu'une fosse remplie de sang en ébulition. En face de nous, la colossale silhouette d'\arthur{}...

Nous approchons, \secret{555}, pour voir \arthur{} qui nous désigne de ses mains percée, mais où nul sang ne coule, \excalibur{}. \morgane{}, moqueuse et souriant à la manière d'un carnassier. Elle assure que nous, les «idiots», ne savaient pas et qu'elle nous a attiré sur le lieu de notre naissance !! «La où ton (\arthur{}) règne s'est effondré». Nous décidons d'aider d'obéir à \arthur{} et \ailei{} tente de se saisir de l'épée \secret{508} alors que \morgane{} s'écrie «Si vous retirez cette épée, cette Camelot sera engloutie par le Wyrd, et nous avec. Les Précurseurs contrôleront de nouveau le c\oe{}ur de Tuathan. Vous allez défaire tout ce pour quoi l'humanité s'est battue.» Cela ne nous arrête pas, \secret{456} \ailei{} sent la puissance infuser en elle par l'intermédiaire d'\excalibur{}, mais c'est douloureux. \morgane{} ordonne à \gaheris{} de s'interposer \secret{480}. Nous parvenons à le vaincre une première fois mais \morgane{} le relève, encore plus puissant \secret{388}. \secret{444}, \gaheris{} est vaincu une seconde fois, mais nous sommes tous à moitié morts. \secret{24} \morgane{} s'enfuit alors et \ailei{} prend possession d'\excalibur{} \secret{78}. La forteresse commence alors à fondre et la sortie semble toute indiquée, sans qu'\arthur{} ne se décide à nous suivre. \gain{5XP}. \secret{610} puis \secret{612}, sans le \graal{} mais avec \excalibur{}, nous revoici dans le \coeurtuathan{}.

% ****************************************************************************
% ***************************************************************** Chapitre 9 
% ****************************************************************************
\section{Chapitre 9}

(La mort nous a saisi, mais nous repartons de plus belle)

Il est temps de repartir à travers \tuathan{}.

% *********************************
\subsection{Nourriture et Richesse}
De nouveau, il semble qu'il faille se nourrir d'or à la place de nourriture. (6-4:59), nous devons nous faufiler entre des filaments mais nous en touchons (46) un qui nous plonge dans un sommeil presque cauchemardesque ou nous revoyons \cuanacht{} qui attend notre aide \gain{\terror[2]}, puis une cellule de l'hospice insulaire... \gain{«4° Exploration de Tuathan», 1XP, \texttt{quête}}. (1-2:16) Nous repassons par l'escalier ou nous voyons notre dos devant nous ce qui terrifie ceux moins habitué à la magie \gain{\texttt{quête}, 1XP si 3 en Spiritualité}.

% ********************************
\subsection{Faiblesses exacerbées}
Comme précédemment, nos propres faiblesses sembles exacerbées par Tuathan. Nous traverson une champignonière (1-3:20) où nous nous reponsons sur la mousse (46) \gain{\nrj[1]}, sans nous attarder \gain{\nrj[-2], \texttt{quête}}. Une altercation avec une créature du wyrd (6-6 puis 2-5) nous incite à aller voir plus loin \gain{\texttt{quête}}.

\subsection{Terreur et Vie}
De nouveau, lors du repos, les pertes de \life{} et \terror{} sont inversées. Dans ce quartier, (1-1:6), les murs sont constitués de vers grouillants, ce qui est affreusement angoissant \gain{\terror[3], \texttt{quête}}. Il faut traverser une membrane tendue au travers de la rue pour aller plus loin (5-4:27), les autres peaux nous parlent, une sensation destabilisante (12) \gain{\terror[1] par Spiritualité, 2 \texttt{quêtes}}.


% ***********************************
\subsection{Epuisé ou la vie s'en va}
Après avoir constaté que la \resolutionorrin{} avait ``bougé'' (elle est introuvable), nous ressentons de nouveau cette impression que si le soir ne nous trouve pas épuisé, notre force vitale est aspirée au loin. Nous sommes maintenant dans un véritable labyrinthe (3-1:13) avec une fontaine dont nous touchons l'eau (32), pour nous retrouver brusquement de l'autre côté, grandement perturbés \gain{\terror[3], \texttt{quête}}.

% ****************
\subsection{Folie oubliée}
Enfin, nous retrouvons cet état de stase qui nous débarasse de notre peur de la folie. Bientôt entourés de gens perdus qui essaient de se comporter comme des humains. Nous parvenons à les gérer \gain{1XP, \texttt{quête}} avant de gagner une vaste forêt (2-4:30) d'où émerge une féroce créature, \gain{\texttt{quête}}.

Notre cheminement dans le Wyrd de \tuathan{} a duré bien trop longtemps \secret{666}. C'est à genoux que nous atteignons difficilement la \valleegardiens{}, malheureusement gardée par un puissant Précurseur. Après une courte prière, nous tentons de discuter avec lui \secret{696}. Comme il admet que nous ne sommes pas une menace \secret{530}, il nous épargne. Et nous nous réveillons dans une petite infirmerie, veillés par \orrin{}. \gain{perte tous objets sauf 2, \life[4], \terror[-4], \nrj[] au max; «Vent du Wyrd»} plus de ``repos troublé''\secret{67}.

\secret{7} En regagnant doucement nos forces, nous réalisons que nous avons été absents un mois. \secret{2} la guerre a fait rage dans l'ouest et \mordred{}, Seigneur de \nidcorbeaux{} a rasé \boismuraille{} et \loincomtat{} \gain{«2° Egarés et Déchus», «3° Egarés et Déchus»}. \secret{5} puis \secret{9}, \cuanacht{} est au bords de l'extinction, \camelot{] nous prie de venir rapidement alors qu'une cité de l'ouest nous propose une mission très lucrative (ce qui paraît difficile maintenant qu'elles ont été rasées ??). Mais, dit \orrin{}, le voyage sera difficile, voir impossible, car ``de très nombreux Menhirs brisés et le Wyrd a envahi une partie du Royaume'' \gain{retirer lieux \cuanacht{}, \tumulus{}, \lacmiroir{} et \newcamelot{}}. Cependant, \secret{3}, \orrin{} nous apprend que la réalité, notre monde, est le lieu où se rendent les Précurseurs destinés aux enfers. Et le Menhirs permettent d'observer ce monde, Menhir dont \merlin{} et \arthur{} ont augmenté la puissance pour que les lois de notre monde aient aussi cours autour d'eux. Il est ainsi possible de construire quelque chose qui permet de voyager dans le Wyrd \gain{«Enigme du Vieiacier» si on ne l'avait pas}.

% ****************************************************************************
% **************************************************************** Chapitre 10 
% ****************************************************************************
\section{Chapitre 10}

Le savoir partagé par \orrin{} nous permettrait de construire des sortes de Menhirs, mais pour cela il faudrait des outils spéciaux, les \outilspierre{}. Alors que nous cherchons comment nous allons pouvoir les trouver, \beor{} se souvient qu'il les détient depuis... longtemps, enfin depuis qu'une vieille les lui a donné à \loincomtat{}. Bon, c'est déjà ça.

Il nous faut donc trouver du \wyrdroche{} le plus pur possible, il semble qu'il y en ait du côté de \loincomtat{}, justement. Alors nous voilà partis.

\questOK{Trouve une \wyrdparfait{}}{Les anciennes mines à côté de \loincomtat{} pourraient encore en contenir.}{Trouvé à \devastation{}}

Nous quittons donc \longcairn{} après avoir combattu un Mur d'Âmes Maudites qui nous tombe dessus depuis la \valleegardiens{}. Arrivés au \visageserein{}, on pousse jusqu'au \cerclelunaire{}. Là, un voyage au centre du cercle nous montre combien nos problèmes sont futiles et qu'il est aisé de les envisager sous un nouveau jour (\gain{-2XP, \nour[3] ou \rich[1] ou \magic[3] pour changer de \competence{}}. Après une visite à l'archidruide \amergin{}, il nous indique que nous n'avons pas encore assez de puissance (il faudrait le \secret{65} avec un cadran sur 5). Au niveau des masures (1), un Druide puis la foule insultent \larve{}, qualifié de traître \secret{211}. Mais \larve{} sait montrer qu'il a changé \secret{102} pour être finalement accueilli comme le fils prodigue \gain{ «Rédemption de Larve», 2XP}. Enfin, nous allons arbitrer la succession de \geraint{} en dénonçant les agissements passés de \morfran{} (esclavagiste) ou \breagach{} (faussaire), ce qui promeut la cousine de \geraint{} comme pouvant lui succéder \gain{«6° Restaurer l'Ordre», 2XP}.

Un aller retour infructueux vers la \croisee{} et nous voilà au \valsanglant{}. Le champ de bataille est devenu silencieux, il ne reste que des charognards. On décide de traverser le champ de bataille \gain{\life[-1]}, ce qui nous fait affronter une créature de cauchemar \gain{\rich[1]}. Petite halte \sousmuraille{} pour chasser, et nous voilà arrivé à \loincomtat{} que le Wyrd a envahi. Il n'en reste plus qu'un \dolmeneffondre{} et une longue colonne de réfugiés. Nous essayons de trouver le grand hall de la reine \kincaid{}, il faut creuser pour en déblayer un accès,ce qui demande de l'énergie et de convaincre les habitants. On y sauve une jeune fille, \siobhan{}, la fille de la Reine qui s'enfuit en jurant de se venger (mais on ne sait pas de qui). Avond nous libéré un démon ? \gain{\terror[2], 2XP, \rich[5], «7° Egarés et Déchus»}.

Nous atteigons \devastation{} sous une pluie torrentielle. C'est là, dans les profondeurs des mines (8+) que nous atteignons les galeries anciennes dont il possible d'extraire un morceu de \wyrdparfait{} \secret{158}. Reste maintenant à le ramner à \orrin{}, ce qui ne sera pas une mince affaire.

\questOK{Ramener la \wyrdparfait{}}{\orrin{} l'attend à \longcairn{}}{Fait, lentement}

Sous un temps magnifique, nous nous attelons à la tâche. En passant d'abord par le  \valsanglant{}, \visageserein{} et enfin \longcairn{}. Là, \orrin{} façonne doucement une silhouette encapuchonnée en nous expliquant que notre réalité exerce une fascination sur les Précurseurs, une sorte d'enfer qu'ils ont voulu observer par le biais des Menhirs. L'\oe{}uvre terminée, \orrin{} effrayé annonce qu'il y a un prix, lourd, à payer : l'essence d'un être vivant. Nous nous proposons \secret{571}, mais \orrin{} explique que notre essence vital est ``particulière'', insoumise et impossible à soumettre, elle finirait par quitter la pierre \gain{\terror[1], 1XP}... \orrin{} se sacrifie donc \secret{533} en nous recommandant d'user de notre (grand) pouvoir à bon escient, ne pas répéter le passé, aller de l'avant. \gain{\terror[1], 2XP, \menhirrudiment{} \secret{15}}.

% ****************************************************************************
% **************************************************************** Chapitre 11
% ****************************************************************************
\section{Chapitre 11}

Toujours à \longcairn{}, il nous faut choisir un avenir entre 1) aider notre village natal de \cuanacht{}, 2) aider \newcamelot{} ou 3) aider nos ``alliés'' (même si nous ne savons pas trop qui sont nos alliés). Ne sachant pas trop que faire, nous décidons d'aller voir \mordred{} à \nidcorbeaux{}. Avec le beau temps, on voyage rapidement jusqu'à \croisee{} d'où nous utilisons notre menhir portable pour révéler \nidcorbeaux{}. \beor{}, parti seul, va scruter la Pierre des Rancunes mais n'y gagne que \gain{\terror[1]} et pas grand chose qui nous permettrait d'apaiser la guerre en avalon. Il remarque néanmoins (7) qu'\aelisa{}, une célèbre marchande de \boismuraille{}, a fait appel à des mercenaires pour éliminer ses partenaires commerciaux, mercenaires qu'elle a ensuite dupé. Il y a de quoi là organiser un beau chantage \gain{ «3° Pillard», \secret{9} ``Informations confidentielles''}. \beor{} se dirige ensuite vers les huttes de pierre où il aide les réfugiés (\gain{\rep[5]\index{Gain!Reputation}, \rich[-5]} et visite les bâtiments commerciaux (3) \gain{\rich[1]\index{Gain!Richesse}, \nrj[-1]}.

On arrive à \nidcorbeaux{} sous une pluie torrentielle. Nous allons voir \mordred{} qui nous reçoit bien que nous ne soyons pas des alliées d'Avalon (17). Pour ancrer son pouvoir dans cette partie de l'île, il souhaiterait les grimoires de sa mère \morgane{}, ce qui laisserait Camelot sans défense. \gain{«5° Eclaireur»}.

\quest{Récupérer les grimoires à \newcamelot{} ou rejoignez une des autres puissances en Avalon.}

Nous passons ensuite au marché (10) pour faire le plein de provision \gain{\rich[-1], \nour[1]\index{Gain!Nourriture}}.

Bon ! Allons voir comment se porte \cuanacht{}...

On redescend vers le sud, en passant par la \croisee{}, le \cerclelunaire{}, le \visageserein{} et le \valsanglant{}. De là, nous révélons le \tumulus{} pour s'apercevoir que son menhir à disparu, englouti comme le tumulus. On appelle maintenant ce lieu le \tumulusengloutis{} \trade{e1v1>M1}. Comme le temps s'améliore, on rejoint le \bosquet{} pour arriver à un \cuanachtdevaste{}. Les quelques dizaines de personnes qui sont encore là nous demandent de l'aide, et nous acceptons (1). Il faut pour cela trouver deux endroits sûrs (6) et \secret{777} \gain{+2XP, \secret{4} ``Enragé'' et \secret{42} ``Méfiance''}.

% ****************************************************************************
% **************************************************************** Chapitre 13 
% ****************************************************************************
\section{Chapitre 13}

Au moment de quitter \cuanachtdevaste{}, nous faisons une rencontre inattendue en la personne du Roi \arthur{}. Il semble se diriger vers \camelot{}, mais pour quel dessein ?

\quest{Destin final d'Arthur}{Que va faire \arthur{}?}

Pour l'heure, nous nous dirigeons vers le sud et les \lames{}. Cette fois, nous ggrimpons tout en haut des gigantesques épées, pour y découvrir que des feuilles d'or couvrent en partie le pommeau \gain{1XP, \rich[2], «1° trésors cachés»}. Ensuite, \beor{} fabrique un bouclier en osier. Avec la nuit, nous rêvons de la première bataille, où \arthur{} fit mettre un genoux à terre au géants \gain{\terror[-1]\index{Gain!Terreur}}.

Le temps se maintient au beau, on pousse jusqu'à \larvebois{}, un endroit peu agréable qui grouille de larves blanches. Nous cherchons à atteindre le c\oe{}ur de la forêt, pour y découvrir un précurseur bien diminué. Il nous demande d'aller quérir son anneaux, que l'on décide de garder en éliminant la créature. Rêver, ou cauchemarder ici n'est pas de tout repos...

Le temps est magnifique, on remonte vers le \bosquet{} puis \blanc{} sous une lune magnifique. Les habitants sont encore plus malades qu'avant (14-15) \gain{«4° remède»} mais acceptent de nous aider (7-12-4) en nous offrant un objet. Le lendemain nous voit chasser à \tordracine{} avant d'arriver au \lacsec{} dans la tempête.

Au centre du lac d'où l'eau s'est retirée, une épine de wyrdroche abrite une grotte labyrinthique. On y trouve la \dame{}, fatiguée et alitée. Elle pense qu'il existe peut-être un endroit sûr le long de la côte, du côté des bateaux (?) mais, plus sûrement, vers le \visageserein{}, ce qui demandera de briser les sceaux \gain{«6° rêves et prophéties»}.

\questOK{Visage Serein}{Aller au \visageserein{} et le transformer en Refuge.}{Brise les sceaux dans la salle aux étoiles}

Après nous être débarassé d'un autre gardien, nous prenons donc la route du \visageserein{} sous un beau temps. Une fois sur place, nous pénêtrons dans le visage par son oreille. Suivre nos anciennes traces ne révèle rien de nouveau, alors on se glisse dans une fissure (3) pour arriver à une grande salle vide où brillent les étoiles (7), des milliers de pierres précieuses enchassées dans la voute. Il est possible d'y entrer en transe, mais nous préférrons trouver les sceaux indiqués par la \dame{} (13). Plusieurs jours sur place, à se reposer sous le dôme \gain{\magic[2]]\index{Gain!Magie}, \terror[-2]\index{Gain!Terreur}} seront nécessaires avant de pouvoir effectivement briser les sceaux \gain{«5° Dernier Refuge»}.

Nous retournons alors au \bosquet{} pour chasser, puis poussons jusqu'à \blanc{}. Il nous faut affronter un chevalier errant avant d'arriver à la \flotte{}. Nous y fouillons les épaves (7) pour trouver de l'or \gain{\rich[2], «5° Trésors cachés»}. Dans le fortin (2) abandonné, des traces de passages récents et la herse a été entaillée à plusieurs endroits. On se résoud alors à nager jusuq'à l'arbre (5); grâce à Excalibur \secret{78}, \ailei{} peut briser le dôme invisible qui protège la flotte \gain{2XP, «3° Dernier Refuge»}.

Nous avons donc au moins 2 refuges \secret{611}, il est temps de réfléchir à la suite...

% ****************************************************************************
% **************************************************************** Chapitre 14 
% ****************************************************************************
\section{Chapitre 14}

Il est clair qu'il faut mener des communautés aux refuges que nous avons trouvés ou fabriqués.

\questOK{Amener une communauté en sécurité}{Si nous réussisson, nous pourrons obtenir le \secret{622}}{De \cuanachtdevaste{} à \flotte{}}.

On retourne donc vers \blanc{}, en croisant un noble offensé. \beor{} reste ``diplomate'' avec les habitants et ne casse que quelques os... Par beau-temps, on pousse jusqu'à \cuanachtdevaste{} en passant par le \bosquet{}. Là, nous nous proposons de guider la population vers un endroit sûr (11) \gain{«1° Abandonnes», «7° Confrontation Finale»}, les colons remplis d'espoir \secret{33} nous suivent.

Nous les faisons passer par le \premierfort{} sous un temps magnifique avant d'arriver à la \flotte{}. Les survivants prennent place à bord (3) de deux douzaines de navires \gain{«3° Vestiges», \rep[3], +3XP}. Mais, comme nous avons encore des choses à faire (nous voulons savoir ce qu'il est advenu d'\arthur{}), nous ne partons pas avec eux. Nos aventures ne sont pas finies.









  
















\printindex

\end{document}
