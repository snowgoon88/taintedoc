\documentclass[11pt]{article}

\usepackage[french]{babel}
\usepackage[utf8x]{inputenc}
\usepackage[OT1]{fontenc}
%\usepackage{fullpage}
\usepackage[paperheight=73cm, paperwidth=52cm,top=0.5cm, bottom=0.5cm, left=0.5cm, right=0.5cm, showframe]{geometry}
\usepackage{hyperref}
\usepackage{marginnote}

%% \usepackage{tensor}
%% \usepackage{enumitem}
%% \usepackage{outlines}
%% \usepackage{ulem}

\usepackage{makeidx}
\usepackage{xparse}

\usepackage{fontawesome5}
\usepackage{awesomebox}
\usepackage{xcolor}
\usepackage[most]{tcolorbox}
\usepackage{amssymb}
\usepackage{bm} %boldmath
\usepackage{soul}

\usepackage{graphicx}
\usepackage{tikz}
\usetikzlibrary{shapes.geometric}
%% Activate externalization to separate PDF for every image
%% \usetikzlibrary{external}
%% \tikzexternalize 

% ****************************************************************************
% ********************************************************************* COMMON 
% ****************************************************************************
%% *****************************************************************************
%% ********************************************************************* questOK
%% *****************************************************************************
\newtcolorbox{questbox}{colback=violet!10,colframe=white,left skip=0.2\textwidth}
\newcommand{\quest}[2]{%
  {\footnotesize
    \begin{questbox}{\hfill\textbf{#1}\vspace{2mm}\hrule}
    \begin{awesomeblock}[violet]{2pt}{\faQuestionCircle[regular]}{violet}
      #2
    \end{awesomeblock}
  \end{questbox}
  }
}

\newtcolorbox{questOKbox}{colback=green!10,colframe=white,segmentation style={draw=violet,solid},left skip=0.2\textwidth}
\newcommand{\questOK}[3]{%
  {\footnotesize
  \begin{questOKbox}{\hfill\textbf{#1}\vspace{2mm}\hrule}
    \begin{awesomeblock}[green]{2pt}{\faQuestionCircle}{green}
      {
        #2

        {\color{green}\vspace{2mm}\hrule\vspace{2mm}}

        $\leadsto$ #3
      }
    \end{awesomeblock}
  \end{questOKbox}
  }
}
%% *****************************************************************************
%% *****************************************************************************
%% *****************************************************************************
\newcommand{\nrj}[1][]{{\color{teal} #1\faBolt}}
\newcommand{\nour}[1][]{{\color{olive} #1\faDrumstickBite}}
\newcommand{\rich}[1][]{{\color{lightgray} #1\faCoins}}
\newcommand{\rep}[1][]{{\color{orange} #1\faGrin}}
\newcommand{\magic}[1][]{{\color{violet} #1\faSun}}
\newcommand{\life}[1][]{{\color{red} #1\faHeart}}
\newcommand{\terror}[1][]{{\color{black} #1\faSurprise}}

\newtcolorbox{tradebox}{colback=yellow!10,colframe=white}
%% *****************************************************************************
%% Define a macro parsing its input argument one character at a time

%% needed in Tex to be used in macro name
\makeatletter

\newcounter{@tradeitems}
%% print ', ' if already an item before
\newcommand\@tradeseparator{%
  \ifnum\value{@tradeitems}>0 {,} \fi
  \addtocounter{@tradeitems}{1}
}

%% e.E: Energie, n/N: NOUR, o/O: RICH, r/R: REP, m/M: MAGIC
%% \trade[option put after]{combinaison de eXnXoXrXmX > EXNXOXRXMX}
\newcommand\trade[2][]{%
  \setcounter{@tradeitems}{0}
  %\marginnote[right]{[\@mytrade#2\@nnil#1]}
  [\@mytrade#2\@nnil#1]
}

\def\@mytrade#1{%
  %% if #1 == \@nil do nothing and stop expanding
  %% sinon utilise \mymacro@char@#1 et continue la suite
  \ifx\@nnil#1\relax\else
  \@nameuse{mytrade@char@#1\expandafter}%
  \fi
}
%% Définit un "faiseur de macro"
%% \defcharcode#1 définira une macro \mytrade@char@#1 qui finira par
%% appeler récursirvement \@mytrade
\def\defcharcode#1{%
    \@namedef{mytrade@char@#1}%
}
%% e: use NRJ
\defcharcode{e}#1{%
  \@tradeseparator%
  \nrj[-#1]%
  \@mytrade
}
%% n: use NOUR
\defcharcode{n}#1{% reads further character after q
  \@tradeseparator%  
  \nour[-#1]%
  \@mytrade
}
%% o: use OR
\defcharcode{o}#1{% reads further character after q
  \@tradeseparator%
  \rich[-#1]%
  \@mytrade
}
%% r: use REP
\defcharcode{r}#1{% reads further character after q
  \@tradeseparator%
  \rep[-#1]%
  \@mytrade
}
%% m: use MAG
\defcharcode{m}#1{% reads further character after q
  \@tradeseparator%
  \magic[-#1]%
  \@mytrade
}
%% v: use LIFE
\defcharcode{v}#1{% reads further character after q
  \@tradeseparator%
  \life[-#1]%
  \@mytrade
}
%% t: use TERROR
\defcharcode{t}#1{% reads further character after q
  \@tradeseparator%
  \terror[#1]%
  \@mytrade
}
\defcharcode{>}{%
  \setcounter{@tradeitems}{0}
   $\Rightarrow$ 
  \@mytrade
}
%% E: gain NRJ
\defcharcode{E}#1{%
  \@tradeseparator%
  \nrj[#1]\index{Gain!Energie}%
  \@mytrade
}
%% N: gain NOUR
\defcharcode{N}#1{% reads further character after q
  \@tradeseparator%
  \nour[#1]\index{Gain!Nourriture}% 
  \@mytrade
}
%% O: gain OR
\defcharcode{O}#1{% reads further character after q
  \@tradeseparator%
  \rich[#1]\index{Gain!Richesse}%
  \@mytrade
}
%% R: gain REP
\defcharcode{R}#1{% reads further character after q
  \@tradeseparator%
  \rep[#1]\index{Gain!Reputation}%
  \@mytrade
}
%% M: gain MAG
\defcharcode{M}#1{% reads further character after q
  \@tradeseparator%
  \magic[#1]\index{Gain!Magie}%
  \@mytrade
}
%% V: gain LIFE
\defcharcode{V}#1{% reads further character after q
  \@tradeseparator%
  \life[#1]\index{Gain!Vie}%
  \@mytrade
}
%% T: gain Terror
\defcharcode{T}#1{% reads further character after q
  \@tradeseparator%
  \terror[-#1]\index{Gain!Terreur}%
  \@mytrade
}


\makeatother
%% *****************************************************************************

%% *****************************************************************************
%% ****************************************************************** secret/loc
%% *****************************************************************************
\newcommand{\loc}[1]{%
  (lieu #1)
}
\newcommand{\secret}[1]{%
  [secret #1]
}
%% using xparse
%% 1=Nom, [2=entry, default=XX], [3=key pour index sans accent]
\NewDocumentCommand{\lieu}{m O{XX} o}{%
  \underline{#1} (#2)\index{\IfNoValueTF{#3}{#1}{#3@#1}}%
}

\def\adiante{\lieu{Mer d'Adiante}[153]}
\def\blanc{\lieu{Blanc-Lichen}[107]}
\def\boismuraille{\lieu{Bois-Muraille}}
\def\bosquet{\lieu{Bosquet des Chasseurs}[102]}
\def\bourgr{\lieu{Bourg Pestiféré}[138,142][Bourg]}
\def\bourgpacif{\lieu{Bourg Pacifié}[142][Bourg]}
\def\broch{\lieu{Broch Cruach}[136]}
\def\bundorca{\lieu{Bundorca}[135,148]}
\def\camelot{\lieu{Camelot}}
\def\cerclelunaire{\lieu{Cercle Lunaire}[133]}
\def\chatoyantes{\lieu{Étendues Chatoyantes}[157][Etendues Chatoyantes]}
\def\conclave{\lieu{Conclave Calciné}[104][Conclave Calcine]}
\def\croisee{\lieu{Croisée}[152][Croisee]}
\def\cuanacht{\lieu{Cuanacht}[101]}
\def\debacle{\lieu{Debâcle}[118][Debacle]}
\def\faldorca{\lieu{Faldorca}[148]}
\def\falfuar{\lieu{Falfuar}[134,147]}
\def\flotte{\lieu{Flotte d'Épaves}[119][Flotte d'Epaves]}
\def\foire{\lieu{Foire du Guerrier}[103]}
\def\hospice{\lieu{Hospice Insulaire}[109]}
\def\lacmiroir{\lieu{Lac Miroir}[113]}
\def\lames{\lieu{Lames de la Mélancholie}[105][Lames de la Melancholie]}
\def\larvebois{\lieu{Larvebois}[108]}
\def\loincomtat{\lieu{Loincomtat}[116]}
\def\newcamelot{\lieu{Nouvelle Camelot}[190]}
\def\marchestitan{\lieu{Marches du Titan}[115]}
\def\monasteremere{\lieu{Monastère de la Toute-Mère}[140][Monastere de la Toute-Mere]}
\def\murmures{\lieu{Forêt des Murmures}[155][Foret des Murmures]}
\def\nidcorbeaux{\lieu{Nid-de-Corbeaux}[160]}
\def\noirbourbe{\lieu{Noirebourbe}[140]}
\def\plainessacrees{\lieu{Plaines Sacrées}[???][Plaines Sacrees]}
\def\premierfort{\lieu{Premier Fort}[117]}
\def\templemere{\lieu{Temple de la Toute-Mère}[115][Temple de la Toute-Mere]}
\def\tombesordre{\lieu{Tombes de l'Ordre}[154]}
\def\tordracine{\lieu{Tordracine}[114]}
\def\tuathan{\lieu{Tuathan}}
\def\tumulus{\lieu{Tumulus des Précurseurs}[106][Tumulus des Precurseurs]}
%%\newcommand{\perso}[1]{\textbf{#1}}
%% using xparse
%% 1=Nom, [2=key pour index sans accent]
\NewDocumentCommand{\perso}{m o}{%
  \textbf{#1}\index{\IfNoValueTF{#2}{#1}{#2@#1}}%
}
\def\agravain{\perso{Agravain}}
\def\amergin{\perso{Amergin}}
\def\arev{\perso{Arev}}
\def\arthur{\perso{Arthur}}
\def\aubert{\perso{Aubert}}
\def\badb{\perso{Badb}}
\def\bedivere{\perso{Bedivère}[Bedivere]}
\def\beor{\perso{Béor}[Beor]}
\def\bors{\perso{Bors}}
\def\breagach{\perso{Bréagach}[Breagach]}
\def\cosuil{\perso{Cosuil}}
\def\dame{\perso{Dame du Lac}}
\def\dobromir{\perso{Dobromir}}
\def\erfir{\perso{Erfyr}}
\def\fael{\perso{Faël}[Fael]}
\def\gaheris{\perso{Gaheris}}
\def\galaad{\perso{Galaad}}
\def\gauvain{\perso{Gauvain}}
\def\geraint{\perso{Geraint}}
\def\guivrarpion{\perso{Guivrarpion}}
\def\kincaid{\perso{Kincaid}}
\def\larve{\perso{Larve}}
\def\lamorak{\perso{Lamorak}}
\def\lancelot{\perso{Lancelot}}
\def\macha{\perso{Macha}}
\def\mordred{\perso{Mordred}}
\def\morfran{\perso{Morfran}}
\def\morgane{\perso{Morgane}}
\def\neante{\perso{Néante}[Neante]}
\def\nemain{\perso{Nemain}}
\def\orrin{\perso{Orrin}}
\def\palamede{\perso{Palamède}[Palamede]}
\def\yvain{\perso{Yvain}}

%\newcommand{\art}[1]{\texttt{#1}}
\NewDocumentCommand{\art}{m o}{%
  \texttt{#1}\index{\IfNoValueTF{#2}{#1}{#2@#1}}%
}
\def\craie{\art{Craie Rituelle}}
\def\graal{\art{Graal}}
\def\fauxgraal{\art{Faux Graal}}
\def\guivre{\art{Guivre}}
\def\masque{\art{Masque Funéraire}[Masque Funeraire]}
\def\outilspierre{\art{Outils de Tailleur de Pierre}}
\def\pierrerancune{\art{Pierre des Rancunes}}
\def\poupee{\art{Poupée de Paille}[Poupee de Paille]}
\def\tetemorrigan{\art{Tête de Morrigan}[Tete de Morrigan]}

\newcommand{\gain}[1]{{\color{red}\textbf{[#1]}}}

\newcommand{\unk}[1]{{\color{orange}(#1)}}


% ****************************************************************************
% ***************************************************************** TIKZ Macro 
% ****************************************************************************
% name,pos,title, trade, done, todo
\newlength{\cardsize}
\setlength{\cardsize}{3.5cm}
\newlength{\linkhor}
\setlength{\linkhor}{3.5cm}
\newlength{\linkver}
\setlength{\linkver}{3.0cm}

\newcommand{\kzcard}[6]{%
  %\begin{scope}[text width=\cardsize]

  \path[text width=\cardsize] {#2} node[matrix,fill=green!10,rounded corners] (#1)
       {
      #3;\\
      %%\node[draw=black] (title) {#3};\\
      \node (trade) {#4}; \draw (node cs:name=trade,anchor=south west) -- (node cs:name=trade,anchor=south east);\\
      \node (done) {#5}; \\
      \node (todo) {{\color{red}\bm{$\leadsto$} #6}};\\
    };
   %\end{scope}
}
\newcommand{\kzcardwyrd}[6]{%
  %\begin{scope}[text width=\cardsize]

  \path[text width=\cardsize] {#2} node[matrix,fill=blue!10,rounded corners] (#1)
       {
      #3;\\
      %%\node[draw=black] (title) {#3};\\
      \node (trade) {#4}; \draw (node cs:name=trade,anchor=south west) -- (node cs:name=trade,anchor=south east);\\
      \node (done) {#5}; \\
      \node (todo) {{\color{red}\bm{$\leadsto$} #6}};\\
    };
   %\end{scope}
}
%% using xparse
%% 1=Nom, [2=entry, default=XX], [3=key pour index sans accent]
\RenewDocumentCommand{\lieu}{m O{XX} o}{%
  %%\underline{#1} (#2)\index{\IfNoValueTF{#3}{#1}{#3@#1}}%
  \node[draw=black,text width=\cardsize] (title) {\textbf{#1} (#2)};
  %%(title.north) \node[ellipse,draw,above=3mm,text width=8mm,align=center] (num) {#2}
}
\newcommand{\kzcardunk}[6]{%
  %\begin{scope}[text width=\cardsize]

  \path[text width=\cardsize] {#2} node[matrix,fill=yellow!10,rounded corners] (#1)
       {
      #3;\\
      %%\node[draw=black] (title) {#3};\\
      \node (trade) {#4}; \draw (node cs:name=trade,anchor=south west) -- (node cs:name=trade,anchor=south east);\\
      \node (done) {PAS VISITÉ}; \\
      \node (todo) {{\color{red}\bm{$\leadsto$} #6}};\\
    };
   %\end{scope}
}

%% using xparse
%% 1=Nom, [2=entry, default=XX], [3=key pour index sans accent]
\RenewDocumentCommand{\lieu}{m O{XX} o}{%
  %%\underline{#1} (#2)\index{\IfNoValueTF{#3}{#1}{#3@#1}}%
  \node[draw=black,text width=\cardsize] (title) {\textbf{#1} (#2)};
  %%(title.north) \node[ellipse,draw,above=3mm,text width=8mm,align=center] (num) {#2}
}

\newcommand{\addmenhir}[1]{%
  \path (#1.north east) node {\pgfimage[height=1cm]{Images/menhir_icon_tg}};
}
\newcommand{\addmeet}[2]{%
  \filldraw[fill=#2,line width=2pt]  (#1.north west) circle[radius=3mm];
}

\newcommand{\nodeunk}[2]{%
  \path #2 node[fill=yellow!10] (n#1) {#1};
}

\newcommand{\mklinkSN}[2]{%
  \draw[line width=2mm] (#1.south) -- (#2.north);
}
\newcommand{\mklinkNS}[2]{%
  \draw[line width=2mm] (#1.north) -- (#2.south);
}
\newcommand{\mklinkWE}[2]{%
  \draw[line width=2mm] (#1.west) -- (#2.east);
}
\newcommand{\mklinkEW}[2]{%
  \draw[line width=2mm] (#1.east) -- (#2.west);
}


%% %% decorate card using character parsing of argument
%% %% needed in Tex to be used in macro name
%% \makeatletter

%% \def\addmeet#1{%
%%   %% if #1 == \@nil do nothing and stop expanding
%%   %% sinon utilise \mymacro@char@#1 et continue la suite
%%   \ifx\@nnil#1\relax\else
%%   \@nameuse{@decorate@char@#1\expandafter}%
%%   \fi
%% }
%% \def\@decorate@char@M{% Menhir
%%   \path (bouh2.north east) node {\pgfimage[height=1cm]{Images/menhir_icon_tg}};
%% }
%% \makeatother
% ****************************************************************************
% ********************************************************************** title
% ****************************************************************************
\title{%
Une carte d'Avalon
}

\makeindex
%% *****************************************************************************
%% ******************************************************************** document
%% *****************************************************************************
\begin{document}

%% Est-ce que tu peux faire un tableau avec les cartes-région : 1e
%% colonne numéro, 2e nom du lieu, 3e effet de la carte (1 Nour contre
%% 1 Rich par ex), 4e ce qu'on y a fait en très résumé, 5e ce qu'il y
%% reste à faire et pourquoi on ne l'a pas fait. Enfin, si t'as l'envie
%% et le temps.


\begin{tikzpicture}
  \path node[anchor=center] {\pgfimage[width=50cm]{Images/map_tg}};
  %\draw[help lines] (-40,-40) grid[step=10] (40,40);
  

  \kzcard{cuanacht}{(-5,-20)}{\cuanacht{}}{\trade{e1>R1}}{Partir}{Explorer avant de finir hachés menu}
  \addmenhir{cuanacht}

  \kzcard{lames}{(cuanacht.south) ++(0,-2\linkver)}{\lames{}}{\trade[obj]{e1r1>}}{Pris marteau}{}
  \mklinkSN{cuanacht}{lames}
  
  \kzcard{bosquet}{(-2,-12)}{\bosquet{}}{\trade[, r.VERT]{e2>N2}}{Chasse en rêve}{}
  \mklinkNS{cuanacht}{bosquet}

  \kzcard{tumulus}{(bosquet.west) ++(-\linkhor,0)}{\tumulus{}}{\trade[, 6 sur D6?]{e2t1>O1}}{Secret des Menhir}{}
  \addmenhir{tumulus}
  \mklinkWE{bosquet}{tumulus}

  %%\nodeunk{112}{(tumulus.north) ++(0,\linkver)}
  \kzcard{valsanglant}{(tumulus.north) ++(0,3\linkver)}{\valsanglant{}}{\trade{e1r1>O1} ou \trade{e1t1>O1}}{Champ de bataille sans vie}{}
  \mklinkNS{tumulus}{valsanglant}
  
  \kzcard{loincomtat}{(tumulus.west) ++(-\linkhor,0)}{\loincomtat{}}{\trade{o1>M1}}{Dolmen}{voir Dame \kincaid{}}
  \addmeet{loincomtat}{blue}
  \mklinkWE{tumulus}{loincomtat}

  \kzcardwyrd{dolmeneffondre}{(loincomtat.center) ++(0,-1)}{\dolmeneffondre{}}{}{\siobhan{}}{}
  
  \kzcard{sousmuraille}{(loincomtat.north) ++(0,3\linkver)}{\sousmuraille{}}{\trade[, r. Verte]{e2>N1}}{\talismanterni{}, Histoire Loincomtat/Boismuraille}{}
  \addmenhir{sousmuraille}
  \mklinkNS{loincomtat}{sousmuraille}
  \mklinkWE{valsanglant}{sousmuraille}

  %\nodeunk{131}
  \kzcard{boismuraille}{(sousmuraille.north) ++(-1.5\linkhor,2.5\linkver)}{\boismuraille{}}{}{1° expédition}{chef mercenaires,\ultan{},quartier commerçant, diplomatie}
  \mklinkNS{sousmuraille}{boismuraille}
  \addmeet{boismuraille}{blue}
  
  %\nodeunk{139}
  \kzcard{devastation}{(loincomtat.west) ++(-\linkhor,0)}{\devastation{}}{\trade{e2v2>O3}}{Parle avec mineur momifi\'e}{Salle ancienne (chap 10)}
  \mklinkWE{loincomtat}{devastation}

  \kzcard{ondechute}{(devastation.west) ++(-\linkhor,0)}{\ondechute{}}{}{ancien péage, tunnel Tuathan}{}
  \addmeet{ondechute}{violet}
  \mklinkWE{devastation}{ondechute}
  
  \kzcard{foire}{(cuanacht.west) ++(-\linkhor,0)}{\foire{}}{\trade[1XP]{e4v1>}}{Tournois}{}
  \mklinkWE{cuanacht}{foire}
  
  \kzcard{debacle}{(foire.west) ++(-\linkhor,0)}{\debacle{}}{\trade{n1>R1}}{Tombe \yvain{}}{}
  \addmenhir{debacle}
  \mklinkWE{foire}{debacle}
  \mklinkNS{debacle}{loincomtat}
  
  \kzcard{bordwyrd}{(debacle.west) ++(-\linkhor,0)}{\bordwyrd{}}{\trade{o2>V3} ou \trade{m2>V3}}{\orrin{}}{Son voyage a Tuathan}
  \addmenhir{bordwyrd}
  \mklinkWE{debacle}{bordwyrd}

  \kzcard{cornes}{(debacle.south) ++(0,-2\linkver)}{\cornes{}}{\trade{e2>N1}}{veileurs, femme sauvage libérée}{}
  \mklinkSN{debacle}{cornes}

  
  \kzcardunk{larvebois}{(lames.west) ++(-\linkhor,0)}{\larvebois{}}{Explorer}{}{}
  \mklinkSN{foire}{larvebois}
  \mklinkWE{lames}{larvebois}
  
  \mklinkWE{larvebois}{cornes}
  
  \kzcard{conclave}{(cuanacht.east) ++(1.6\linkhor,0)}{\conclave{}}{}{«Rites du Menhir»}{}
  \mklinkEW{cuanacht}{conclave}
  \addmeet{conclave}{gray}

  \kzcard{hospice}{(6,-27)}{\hospice{}}{\trade{e1o1>V3}}{Archidruide \guivrarpion{}}{}
  \mklinkSN{conclave}{hospice}

  \kzcard{blanc}{(conclave.north) ++(1,1.5\linkver)}{\blanc{}}{\trade{e1n1>R1}}{Lichen disparaît}{Voir femme \fael{}, passé \breagach{}}
  \mklinkNS{conclave}{blanc}
  \mklinkWE{blanc}{bosquet}
  \addmeet{blanc}{blue}
  \addmenhir{blanc}

  \kzcard{premierfort}{(9,-7)}{\premierfort{}}{}{\guivre{}, réparation Menhir}{Mieux explorer le camps}
  \mklinkEW{blanc}{premierfort}
  \addmeet{premierfort}{violet}

  \kzcard{flotte}{(9,-21)}{\flotte{}}{\trade[, 1XP]{e4t1>M1}}{Flotte protégée par Merlin}{Remettre la Flotte en état}
  \mklinkSN{premierfort}{flotte}
  
  \kzcard{noirbourbe}{(premierfort.east) ++(1.4\linkhor,-4)}{\noirbourbe{}}{\trade{e2v1>M1}}{}{Y chercher le Monastère de la Toute-Mère ?}
  \mklinkEW{premierfort}{noirbourbe}
  \addmenhir{noirbourbe}

  \kzcard{chatoyantes}{(noirbourbe.east) ++(1.2\linkhor,-1.5)}{\chatoyantes{}}{\trade{e2>T1}}{Récupère Poupée}{Autre Relique (masque)?}
  \mklinkEW{noirbourbe}{chatoyantes}
  
  \kzcard{tordracine}{(3,-2)}{\tordracine{}}{\trade[, r.VERT]{e2>N2}}{\fael{}, \fauxgraal{}, \cosuil{}}{}
  \mklinkNS{blanc}{tordracine}

  \kzcard{lacmiroir}{(tordracine.west) ++(-\linkhor,0)}{\lacmiroir{}}{\trade{>E1T1V1}}{\dame{} pas invoquée}{}
  \mklinkWE{tordracine}{lacmiroir}
  \mklinkSN{lacmiroir}{bosquet}

  \kzcard{marchestitan}{(tordracine.east) ++(0.9\linkhor,1)}{\marchestitan{}}{Explorer}{Temple de la Toute-Mère, \gauvain{} blessé}{}
  \mklinkEW{tordracine}{marchestitan}
  \mklinkSN{marchestitan}{premierfort}
  \addmenhir{marchestitan}

  \kzcard{bourgpacif}{(14,0)}{\bourgpacif{}}{}{Pacifie avec \gaheris{}, info sur 1° Expédition}{}
  \mklinkEW{marchestitan}{bourgpacif}
  \addmeet{bourgpacif}{blue}
  \filldraw[fill=gray,line width=2pt] (bourgpacif.north west) ++(3mm,0) arc[radius=3mm, start angle=0, end angle=180] -- cycle;

  \kzcard{newcamelot}{(bourgpacif.east) ++(1.5\linkhor,-3)}{\newcamelot{}}{\trade[1 XP]{e2>}}{\lancelot{} et pb de Table Ronde, \morgane{} récupère Poupée}{Aider ou Détruire Table Ronde ?}
  \mklinkEW{bourgpacif}{newcamelot}

  \kzcard{broch}{(bourgpacif.north) ++(3,2\linkver)}{\broch{}}{\trade[, r.VIO]{e2>O1}}{Rencontres oniriques, passé de \morfran{}}{autres rencontres, passé \morfran{}}
  \mklinkNS{bourgpacif}{broch}

  \kzcard{faldorca}{(9.5,7.5)}{\faldorca{}}{\nrj[5] tracter Menhir}{Rapatrier tous les habitants}{Trouver Havre pour habitants car famine}
  \mklinkWE{broch}{faldorca}
  \addmeet{faldorca}{blue}

  \kzcard{falfuar}{(faldorca.west) ++(-1.5\linkhor,1)}{\falfuar{}}{\nrj[5] tracter Menhir}{Habitant vers Faldorca}{}
  \mklinkWE{faldorca}{falfuar}
  %% passage secret
  \draw[line width=2mm,dashed,->,gray] (falfuar.south) to (tordracine.north);

  %% connecting 2 Menhirs
  \draw[line width=2mm, dotted, color=gray] (faldorca.north east) to [out=135, in=45] (falfuar.north east);
  \addmenhir{faldorca}
  \addmenhir{falfuar}

  \kzcard{murmures}{(18.5,18.5)}{\murmures{}}{\trade[, r.VERT]{e2>N2}}{}{Chercher \neante{}, explorer Carac$\,>3$}
  \mklinkNS{broch}{murmures}
  \addmenhir{murmures}
  
  \kzcard{tombesordre}{(faldorca.north) ++(0,2\linkver)}{\tombesordre{}}{\trade[Tombe]{m2r2>}}{Rend Fourraux Excalibur}{Faire parler \bedivere{} et \palamede{}, Fond du Puit}
  \mklinkNS{faldorca}{tombesordre}
  \mklinkWE{murmures}{tombesordre}

  \kzcard{cerclelunaire}{(-5,8)}{\cerclelunaire{}}{\trade{e2o1>M1} ou \trade{e2n1>M1}}{Dévoile \fauxgraal{}}{Succession de \geraint{}}
  \mklinkWE{falfuar}{cerclelunaire}

  %\nodeunk{132}
  \kzcard{visageserein}{(cerclelunaire.west) ++(-1.5\linkhor,-1.5)}{\visageserein{}}{\trade{e3>T2M1}}{Mystère résolu}{\secret{33} et «5° Dernier Refuge»}
  \mklinkWE{cerclelunaire}{visageserein}
  \mklinkSN{visageserein}{valsanglant}
  \mklinkWE{visageserein}{boismuraille}
  \addmenhir{visageserein}

  \kzcard{longcairn}{(visageserein.north) ++(0,2\linkver)}{\longcairn{}}{\trade[-1 obj]{e1>O1R1}}{1er expedition}{recueillir tombes, \textbf{Tuathan}, ...}
  \mklinkNS{visageserein}{longcairn}

  \kzcard{valleegardiens}{(longcairn.west) ++(-2\linkhor,3)}{\valleegardiens{}}{Explorer}{\aubert{} mort}{}
  \mklinkWE{longcairn}{valleegardiens}

  \kzcard{croisee}{(-4,17)}{\croisee{}}{}{Légendes qu'on aurait vécue ??}{\pierrerancune{}, troc}
  \mklinkNS{cerclelunaire}{croisee}
  \addmenhir{croisee}
  \addmeet{croisee}{blue}

  \kzcard{adiante}{(croisee.east) ++(\linkhor,1)}{\adiante{}}{\life[-1] / rien}{\graal, \dame{} nous protège}{Autres Tréfonds ?}
  \mklinkEW{croisee}{adiante}
  \mklinkSN{adiante}{falfuar}


  \kzcard{nidcorbeaux}{(-7,27)}{\nidcorbeaux{}}{\trade{v1>O1}}{Marché \trade{o1>N1}, infor sur \tetemorrigan{}}{voir Mordred ??}
  \mklinkNS{croisee}{nidcorbeaux}
  \addmeet{nidcorbeaux}{blue}
  
\end{tikzpicture}

\end{document}
